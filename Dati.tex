%----------------------------------------------------------------------------------------
%   USEFUL COMMANDS
%----------------------------------------------------------------------------------------

\newcommand{\dipartimento}{Dipartimento di Matematica ``Tullio Levi-Civita''}

%----------------------------------------------------------------------------------------
% 	USER DATA
%----------------------------------------------------------------------------------------

% Data di approvazione del piano da parte del tutor interno; nel formato GG Mese AAAA
% compilare inserendo al posto di GG 2 cifre per il giorno, e al posto di 
% AAAA 4 cifre per l'anno
\newcommand{\dataApprovazione}{Data}

% Dati dello Studente
\newcommand{\nomeStudente}{Francesco}
\newcommand{\cognomeStudente}{Pecile}
\newcommand{\matricolaStudente}{1120771}
\newcommand{\emailStudente}{francesco.pecile.1@studenti.unipd.it}
\newcommand{\telStudente}{3457871996}

% Dati del Tutor Aziendale
\newcommand{\nomeTutorAziendale}{Antonio}
\newcommand{\cognomeTutorAziendale}{Cavestro}
\newcommand{\emailTutorAziendale}{antonio.cavestro@livestory.nyc}
\newcommand{\ruoloTutorAziendale}{}

% Dati dell'Azienda
\newcommand{\ragioneSocAzienda}{LiveStory SRL}
\newcommand{\indirizzoAzienda}{Via Roma 1, Roma (RM)}
\newcommand{\sitoAzienda}{https://www.livestory.nyc/}
\newcommand{\emailAzienda}{info@livestory.nyc}

% Dati del Tutor Interno (Docente)
\newcommand{\titoloTutorInterno}{Prof.}
\newcommand{\nomeTutorInterno}{NomeDocente}
\newcommand{\cognomeTutorInterno}{CognomeDocente}

\newcommand{\prospettoSettimanale}{
     % Personalizzare indicando in lista, i vari task settimana per settimana
     % sostituire a XX il totale ore della settimana
    \begin{itemize}
        \item \textbf{Prima Settimana}
        \begin{itemize}
            \item Analisi dei requisiti funzionali del sistema da             sviluppare;
            \item Ricerca, studio e documentazione sulle tecnologie e le tecniche di integrazione dati relative a ERP SAP e Google Analytics.
            \item Documentazione relativa.
        \end{itemize}
        \item \textbf{Seconda Settimana} 
        \begin{itemize}
            \item Progettazione base dati MongoDB;
            \item Progettazione e implementazione del software di interrogazione e estrazione dati da SAP e/o Google Analytics;
            \item Documentazione relativa.
        \end{itemize}
        \item \textbf{Terza Settimana} 
        \begin{itemize}
            \item Progettazione architetturale di alto livello dell’applicativo web;
            \item Definizione di interfacce (REST);
            \item Approfondimento tecnologico;
            \item Documentazione relativa.
        \end{itemize}
        \item \textbf{Quarta Settimana} 
        \begin{itemize}
            \item Progettazione di basso livello dell’applicativo web;
            \item Inizio implementazione applicativo;
            \item Documentazione relativa.
        \end{itemize}
        \item \textbf{Quinta Settimana} 
        \begin{itemize}
            \item Prosecuzione dell’implementazione dell’applicativo e scrittura relativa documentazione;
            \item Rilascio in ambiente di sviluppo della prima base di applicazione;
            \item Documentazione relativa.
        \end{itemize}
        \item \textbf{Sesta Settimana } 
        \begin{itemize}
            \item Prosecuzione dell’implementazione applicativo web e scrittura relativa documentazione;
            \item  Rilascio in ambiente di sviluppo.
        \end{itemize}
        \item \textbf{Settima Settimana} 
        \begin{itemize}
            \item Conclusione implementazione applicativo web e relativa
documentazione;
            \item Milestone: completamento obiettivi minimi.
        \end{itemize}
        \item \textbf{Ottava Settimana} 
        \begin{itemize}
            \item Ricerca, studio e documentazione relativa a pattern microservices,DevOps e continuous integration. (formativo);
            \item Documentazione relativa.
            \item Implementazione del supporto di Docker da parte dell’applicativo web sviluppato. (produttivo);
            \item Milestone: completamento obiettivi massimi.
        \end{itemize}
    \end{itemize}
}

% Indicare il totale complessivo (deve essere compreso tra le 300 e le 320 ore)
\newcommand{\totaleOre}{320}

\newcommand{\obiettiviFormativiMinimi}{
	 \item \underline{\textit{FMI1}}: Conoscenza dello stack tecnologico basato su Node.js, MongoDB,
Mongoose e AngularJS, peculiarità, pro e contro. Quali sono i possibili
casi d’uso e quali no;

	 \item \underline{\textit{FMI2}}:Esplorazione e studio di soluzioni di system integration verso ERP SAP
B1 e/o al servizio Google Analytics;
	 \item \underline{\textit{FMI3}}:  Esplorazione del modello DevOps orientato a rilasci frequenti e
automatizzati, non solo SW Engineering & QA ma anche IT Ops.
	 
}

\newcommand{\obiettiviFormativiMassimi}{
	 \item \underline{\textit{FMA1}}:Presa di coscienza di come il servizio sviluppato si colloca in un
ecosistema di altri servizi. Introduzione ai microservices;
	 \item \underline{\textit{FMA2}}: Avvicinamento a problemi e soluzioni relativi alla scalabilità di sistema
	 \item \underline{\textit{FMA3}}:Perché containers? Avvicinamento alla piattaforma Docker.
}

\newcommand{\obiettiviProduttiviMinimi}{
	 \item \underline{\textit{PMI1}}: Realizzazione base dati in MongoDB per la persistenza dei dati
interessati;
	 \item \underline{\textit{PMI2}}: Realizzazione di un modulo di system integration per il recupero dati da
SAP e/o Google Analytics;
	 \item \underline{\textit{PMI3}}: Realizzazione di un servizio backend e uno frontend per la presentazione
dei dati estratti.

}
\newcommand{\obiettiviProduttiviMassimi}{
 \item \underline{\textit{PMA1}}:Progettazione e sviluppo di servizi orientati alla scalabilità
  \item \underline{\textit{PMA2}}:Progettazione e sviluppo di servizi adatti ad un modello DevOps(Docker);
   \item \underline{\textit{PMA3}}:Dati gli strumenti adeguati, autonomia nello sviluppo e nei rilasci (Assaggio di DevOps e Continuous Integration)
}